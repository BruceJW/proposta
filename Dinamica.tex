% !TEX encoding = UTF-8 Unicode
\documentclass[a4paper, 12pt]{article}
\usepackage{graphicx}
\usepackage{amsmath}
\usepackage{amsfonts}
\usepackage[brazil]{babel}
\usepackage[utf8]{inputenc}
\usepackage{url}

\author{Bruno Giovanini}
\title{ROS}

\linespread{1.5}


\begin{document}
%Capa
\begin{large}


\textbf{Dinâmica}

\vspace{1.5cm}


\end{large}



\newpage


\section{Introdução}

Geralmente, pode ser dito que a dinâmica de um helicóptero é não linear, com o acoplamento de cada eixo. Porém, para voos de baixa velocidade, isto é, velocidades abaixo de 5 $m/s$, a dinâmica pode ser expressa por um conjunto de equações lineares de movimento como um sistema SISO (\textit{single input single output}). A dinâmica do helicóptero é dividida em vários componentes. Deriva-se o modelo para cada componente tanto pela relação geométrica quanto pela equação de movimento. Combinando todos os componentes, é possível extrair duas equações lineares de estado que descrevem os movimentos laterais e longitudinais do helicóptero. Os parâmetros do modelo são determinados por suas especificações \cite{Nonami2010}.

É conhecido que o modelo de atitude de rolagem é o mesmo da afagem devido à simetria do quadricóptero. Portanto, a uma modelagem é desenvolvida da mesma forma para rolagem e afagem.

Devido ao fato da atitude do quadricóptero ser controlada de acordo com a velocidade de rotação do motor, considera-se inicialmente uma expressão que relaciona a atitude e o torque gerado pela diferença da velocidade de rotação do motor no centro do veículo.




\section{Modelo dinâmico de um quadricóptero}

A modelagem do controle de um quadricóptero é um problema complexo e tipicamente requer a existência de um modelo matemático de sua dinâmica. Modelos extremamente complexos que incluem os atuadores e dinâmica dos rotores, aerodinâmica da fuselagem e vibração das hélices não são muito práticos para a maioria dos mecanismos avançados de controle em tempo real. O primeiro passo na direção da modelagem da dinâmica do quadricóptero é considerá-lo como um corpo rígido em tres dimensoes com seis graus de liberdade sujeito a uma força principal e três momentos. Adicionalmente são considerados insignificantes os efeitos dos momentos causados pelo corpo rígido sobre a dinâmica translacional e os efeitos do solo. 

\subsection{Sistemas de Coordenadas}

Para obter e entender o comportamento dinâmico de um quadricóptero, é necessário considerar os seguintes sistemas de coordenadas: inercial, veículo, veículo-1, veículo-2, corpo, estabilidade e vento. O sistema de coordenadas inerciais e do veículo são relacionadas por uma translação, enquanto as demais são relacionadas por rotações. Os ângulos definindo as orientações relativas aos sistemas de coordenadas do veículo, veículo-1, veículo-2 e corpo são os ângulos de rolagem, afagem e guinada, respectivamente, que descrevem a atitude do veículo. Estes ângulos são comumente chamados de Ângulos de Euler. A relação entre os demais sistemas de coordenadas são chamados ângulos de ataque e sideslip, que consideram a velocidade do ar em um ambiente com incidência de ventos, fato muito improvável em ambientes \textit{indoors} e considerado como ruído neste trabalho. 

\subsubsection{Sistema de coordenadas inerciais $\mathcal{F}^i$}

É o sistema de coordenadas fixado à Terra com sua origem definida como o ponto de partida do veículo. Esse sistema, às vezes, é chamado de norte-leste-sul, com a sigla em inglês (\textit{NED}), com o Norte sendo considerado a direção inercial $X$, representado por $i_i$, o leste como a direção inercial $Y$, representado por $j_i$, e o Sul como a direção inercial $Z$, representado por $k_i$, como ilustrado na Figura ???. 

Figura ???

\subsubsection{Sistema de coordenadas do veículo $\mathcal{F}^v$}

A origem do sistema de coordenadas do veículo é o centro de massa do MAV. Porém, os eixos do $\mathcal{F}^v$ são alinhados com o referencial inercial $\mathcal{F}^i$. Em outras palavras, os eixos de $\mathcal{F}^v$ são apenas transladados em relação aos eixos de $\mathcal{F}^i$, possuindo a mesma orientação conforme a Figura ????.

Figura ????

\subsubsection{Sistema de coordenadas veículo-1 $\mathcal{F}^{v1}$}

A origem do sistema de coordenadas veículo-1 é idêntico à origem de $\mathcal{F}^v$. Porém, $k_{v1}$ é rotacionado positivamente na direção da regra da mão direita em relação a $k_v$ pelo ângulo de guinada $\psi$. Os demais eixos $i_{v1}$ e $j_{v1}$ apontam na mesma direção de $i_v$ e $j_v$. O sistema de coordenadas veículo-1 é mostrado na Figura ?????. 

A transformação de $\mathcal{F}^{v}$ para $\mathcal{F}^{v1}$ é dada por:

\begin{equation}
\centering
\mathbf{p}^{v1} = { R }_{ v }^{ v1 }(\psi)\mathbf{p}^{v}
\label{eq:coord1}
\end{equation}

\noindent onde $\mathbf{p}$ indica a posição em relação ao eixo correspondente e: 

\begin{equation}
\centering
{ R }_{ v }^{ v1 }(\psi) = 
\begin{pmatrix} 
\cos { \psi  }  & \sin { \psi  } & 0 \\ -\sin { \psi  }  & \cos { \psi  } & 0  \\ 0  & 0 & 1 
\end{pmatrix}
\label{eq:coord1}
\end{equation}

\subsubsection{Sistema de coordenadas do veículo-2 $\mathcal{F}^{v2}$}

\subsubsection{Sistema de coordenadas do corpo $\mathcal{F}^{b}$}

\subsection{Cinemática}

A velocidade translacional de um MAV é comumente expressada em termos dos componentes das velocidades ao longo de cada eixo do referencial fixado ao corpo do veículo. Já a posição translacional é usualmente medida e expressada no referencial inercial. Portanto, relacionar velocidade e posição translacionais requer diferenciação e uma transformação rotacional como mostrado na equação \ref{eq:cin1}.

\begin{equation}
\centering
\frac { d }{ dt } 
\begin{pmatrix} 
	{ p }_{ x } \\ { p }_{ y } \\ { p }_{ z } 
\end{pmatrix} = 
{ R }_{ b }^{ v }\begin{pmatrix} { u } \\ { v } \\ { w } \end{pmatrix} =
\begin{pmatrix} 
	{ c }_{ \theta  }{ c }_{ \psi  } & { c }_{ \theta  }s_{ \psi  } & -{ s }_{ \theta  } \\ { s }_{ \phi  }{ s }_{ \theta  }{ c }_{ \psi  }-{ c }_{ \phi  }{ s }_{ \psi  } & { s }_{ \phi  }{ s }_{ \theta  }{ s }_{ \psi  }+{ c }_{ \phi  }{ c }_{ \psi  } & { s }_{ \phi  }{ c }_{ \theta  } \\ { c }_{ \phi  }{ s }_{ \theta  }{ c }_{ \psi  }+{ s }_{ \phi  }{ s }_{ \psi  } & { c }_{ \phi  }{ s }_{ \theta  }{ s }_{ \psi  }-{ s }_{ \phi  }{ c }_{ \psi  } & { c }_{ \phi  }{ c }_{ \theta  } 
\end{pmatrix}
\begin{pmatrix}
	{ u } \\ { v } \\ { w } 
\end{pmatrix}
\label{eq:cin1}
\end{equation}

\noindent onde $[p_x,p_y,p_z]^T$ é a posição translacional do veículo em relação ao referencial inercial, ${ R }_{ b }^{ i }$ é a matriz das transformações rotacionais dos ângulos $\psi$, $\theta$ e $\phi$, em sequência, do referencial fixo para o referencial inercial e $[u,v,w]^T$ é a velocidade translacional do corpo em relação ao seu eixo fixo.

Para se realizar o planejamento de trajetória e controle do veículo, normalmente, se faz necessária descrever a dinâmica em relação ao referencial inercial. O referencial inercial é um conjunto de eixos fixados na Terra que é usado como uma referência imutável enquanto o veículo, que possui um referencial fixo, se movimenta. A posição relativa entre tais referenciais varia com o tempo através de rotações e translações. As translações são traduzidas em somas vetoriais e as rotações necessitam de uma forma de representação da orientação do veículo no espaço 3D. Aqui são utilizados os Ângulos de Euler.

\subsubsection{Rep}

Os Ângulos de Euler possibilitam uma representação da orientação 3D de um objeto usando uma combinação de três rotações sobre os diferentes eixos. 

\subsection{Dinâmica de um corpo rígido}

As equações fundamentais que regem o movimento de um corpo rígido partem do Teorema de Conservação do Momento Linear e do Teorema de Conservação do Momento Angular. O primeiro diz que se a força externa total atuando sobre um sistema de partículas é zero, o momento linear total do sistema, o corpo rígido, é conservado. Em outras palavras, a taxa de variação do momento linear é igual a força externa atuando sobre um sistema de partículas. Sua relação pode ser verificada na equação \ref{eq:ml}.

\begin{equation}
\centering
{ \left(\frac { d\mathbf{P} }{ dt }\right)}_{inercial} = \quad{ \mathbf{F} }^{ (e) }
\label{eq:ml}
\end{equation}

\noindent onde $\mathbf{P}$ é o momento linear e ${ \mathbf{F} }^{ (e) }$ denota a resultante das forças externas exercidas no sistema de partículas em um dado instante. Já o segundo diz que o momento angular total de um sistema de partículas se conserva se o torque externo total é nulo. Da mesma forma, a taxa de variação do momento angular é igual ao torque externo total atuando sobre um sistema de partículas em um dado instante. A Equação \ref{eq:ma} demonstra a relação quando considerado o referencial inercial fixo.

\begin{equation}
\centering
{ \left( \frac { d\overrightarrow { L } }{ dt }  \right)  }_{ inercial }=\quad { N }^{ (e) }
\label{eq:ma}
\end{equation}

\noindent onde $\mathbf{L}$ é o momento angular e ${ \mathbf{N} }^{ (e) }$ denota a resultante dos torques externos exercidos no sistema de partículas em um dado instante. 

\subsection{Equações de movimento de Newton-Euler}

O momento angular de um corpo girando em torno de um eixo fixo, em relação a esse eixo, pode ser calculado através do seu momento de inércia $I$ e sua velocidade angular $\omega$ como demonstrado na equação \ref{eq:ma2}.


\begin{equation}
\centering
 \overrightarrow { L } =I\overrightarrow { \omega } 
\label{eq:ma2}
\end{equation}

\noindent substituindo em \ref{eq:ma}, é obtido 

\begin{equation}
\centering
{ \left( \frac { d(I\overrightarrow { \omega  } ) }{ dt }  \right)  }_{ inercial }=\quad { N }^{ (e) }
\label{eq:ma3}
\end{equation}

Quando considerado o referencial inercial em \ref{eq:ma3}, o momento de inércia $I$ varia no tempo, visto que o corpo translada e rotaciona no tempo em relação ao referencial inercial. Porém, quando considera-se um referencial fixado ao corpo rígido, o momento de inércia é independente do tempo, o que torna as equações de movimento sustancialmente mais simples e a equação \ref{eq:ma3} passa a ser definida como:

\begin{equation}
\centering
{ I\left( \frac { d\overrightarrow { \omega  }  }{ dt }  \right)  }_{ corpo }+\quad \overrightarrow { \omega  } \times I\overrightarrow { \omega  } \quad =\quad { N }^{ (e) }
\label{eq:ma4}
\end{equation}

Além disso, quando escolhidos os eixos fixos do corpo como os eixos principais de inércia, o momento de inércia, representado por uma matriz $(I)_{3\times3}$, passa a ser uma matriz diagonal e, considerando as três componentes do sistema separadamente, as Equações de Euler são expressas por:


\begin{equation}
\begin{aligned}
{ I }_{ xx }{ \dot { \omega  }  }_{ x }+{ \omega  }_{ y }I_{ zz }\omega _{ z }-\omega _{ z }I_{ yy }\omega _{ y }\quad =\quad { I }_{ xx }{ \dot { \omega  }  }_{ x } - (I_{ yy } - I_{ zz }){ \omega  }_{ y }{ \omega  }_{ z }\quad &=\quad N_{ x } \\
{ I }_{ yy }{ \dot { \omega  }  }_{ y } - (I_{ zz } - I_{ xx }){ \omega  }_{ z }{ \omega  }_{ x }\quad &=\quad N_{ y }\\
{ I }_{ zz }{ \dot { \omega  }  }_{ z } - (I_{ xx } - I_{ yy }){ \omega  }_{ x }{ \omega  }_{ y }\quad &=\quad N_{ z }
\end{aligned}
\label{eq:eulers}
\end{equation}

\noindent onde $I_1$, $I_2$ e $I_3$ são constantes. É possível observar que as Equações de Euler não são lineares devido aos termos produto $\omega_i\omega_j$. 

Já o momento linear de um corpo se movimentando linearmente em relação ao eixo inercial referencial pode ser calculado através da equação \ref{eq:ml2}.

\begin{equation}
\centering
\overrightarrow { P } = m\overrightarrow { v } 
\label{eq:ml2}
\end{equation}

\noindent onde $m$ é a massa do corpo rígido e $v$ é a velocidade linear do corpo. Substituindo em \ref{eq:ml}, é obtido 

\begin{equation}
\centering
\frac { d(m\overrightarrow { v } ) }{ dt } ={ \mathbf{F} }^{ (e) }
\label{eq:ml3}
\end{equation}

\noindent  transformando-a para o sistemas de coordenadas do corpo rígido:

\begin{equation}
\centering
m\left[ \frac { d\overrightarrow { v }  }{ dt } \quad +\quad \overrightarrow { \omega  } \times \overrightarrow { v }  \right] =\quad { \mathbf{F} }^{ (e) }
\label{eq:ml3}
\end{equation}

\noindent A força translacional $\mathbf{F}^{(e)}$ combina as forças geradas pelos motores, gravidade e outras componentes de força do corpo rígido. Expandindo a equação \ref{eq:ml3} em suas componentes em conjunto com as equações de momento angular, todas em relaçao ao referencial do corpo rígido, obtemos o sistema com seis equações independentes de movimento expresso em \ref{eq:motion}.

\begin{equation}
\begin{aligned}
	m\left[ { \dot { \upsilon  }  }_{ x } - { \omega  }_{ z }\upsilon_{ y } + { \omega  }_{ y }{ \upsilon  }_{ z } \right] \quad &=\quad { F }_{ x } \\ 
	m\left[ { \dot { \upsilon  }  }_{ y } - { \omega  }_{ z }\upsilon_{ x } + { \omega  }_{ x }{ \upsilon  }_{ z } \right] \quad &=\quad { F }_{ y } \\
	m\left[ { \dot { \upsilon  }  }_{ z } - { \omega  }_{ y }\upsilon_{ x } + { \omega  }_{ x }{ \upsilon  }_{ y } \right] \quad &=\quad { F }_{ z } \\ 
	{ I }_{ xx }{ \dot { \omega  }  }_{ x } - (I_{ yy } - I_{ zz }){ \omega  }_{ y }{ \omega  }_{ z }\quad &=\quad N_{ x } \\
	{ I }_{ yy }{ \dot { \omega  }  }_{ y } - (I_{ zz } - I_{ xx }){ \omega  }_{ z }{ \omega  }_{ x }\quad &=\quad N_{ y }\\
	{ I }_{ zz }{ \dot { \omega  }  }_{ z } - (I_{ xx } - I_{ yy }){ \omega  }_{ x }{ \omega  }_{ y }\quad &=\quad N_{ z }
\end{aligned}
\label{eq:motion}
\end{equation}

\subsection{Forças e torques aerodinâmicos}

Para quadricópteros, o mecanismo que gera forças e torques requeridos para controlar seu movimento é realizado por seus quadro motores e hélices. Estes produzem o impulso necessário de forma perpendicular ao plano de rotação dos motores. O impulso principal é gerado ao longo do eixo vertical do corpo e é usado para compensar a gravidade e controlar o movimento vertical do veículo. Os movimentos horizontais no plano ($x$,$y$) são controlados pelo direcionamento do vetor de impulso na direção apropriada, por consequência, resultando em componentes laterais de força. Os torques de controle são, assim, usados para controlar a orientação do corpo do veículo que controla o movimento horizontal.

O quadricóptero pode ser caracterizado por três controles de torque $\tau = (\tau_\phi,\tau_\theta,\tau_\psi)^T$ e uma força principal $F^b = (0,0,u)^T$, onde $\tau_\phi$,$\tau_\theta$ e $\tau_\psi$ são torques de rolagem, afagem e guinada, respectivamente, no centro de massa e $u$ é o impulso total gerado pelos quatro motores controlados de forma independente. O conjunto de equações é ilustrado, em sua forma matricial, em \ref{eq:forca1} e representa  as forças e torques do sistema:

\begin{equation}
\centering
\begin{bmatrix} 
    0 \\ 0 \\ u \\ \tau _{ \phi  } \\ \tau _{ \theta  } \\ \tau _{ \psi  } 
\end{bmatrix}   =
\begin{bmatrix} 0 \\ 0 \\ { { f }_{ 1 }+{ f }_{ 2 }+{ f }_{ 3 }+{ f }_{ 4 } } \\ l({ f }_{ 2 }-{ f }_{ 4 }) \\ l({ f }_{ 1 }-{ f }_{ 3 }) \\ Q_1 + Q_3 - Q_2 - Q_4
 \end{bmatrix}
\label{eq:forca1}
\end{equation}

\noindent onde $l$ é a distância do motor ao centro de massa do veículo e $f_i$ e $Q_i$ $(i=1,2,3,4)$  são os impulsos e torques de rotação das hélices gerados cada motor, respectivamente. 

O impulso $f_i$ e o torque $Q_i$ são geralmente assumidos por serem proporcionais ao quadrado da velocidade angular do motor $w_i$. De fato, as relações entre a velocidade do motor $w_i$ e o impulso $f_i$ e torque $Q_i$ gerados são  muito complexas \cite{McCormick1995}\cite{Prouty1995}. Portanto, o modelo algébrico para geração das forças e torques de controle são aqui descritas da seguinte forma: 

\begin{equation}
\centering
\begin{pmatrix} 
	u \\ \tau _{ \phi  } \\ \tau _{ \theta  } \\ \tau _{ \psi  } 
\end{pmatrix}=
\begin{pmatrix}
	\rho  & \rho  & \rho  & \rho  \\ 
	0 & -l\rho  & 0 & l\rho  \\ 
	-l\rho  & 0 & l\rho  & 0 \\ 
	\kappa  & -\kappa  & \kappa  & -\kappa 
\end{pmatrix}
\begin{pmatrix}
	{ w }_{ 1 } \\ { w }_{ 2 } \\{ w }_{ 3 }\\ { w }_{ 4 }
\end{pmatrix}
\label{eq:forca2}
\end{equation}

\noindent onde $\rho$ e $\kappa$ são constantes positivas caracterizando a aerodinâmica das hélices. As expressões em \ref{eq:forca2} são aproximações válidas que são usadas para movimentos em baixa velocidade. 

Substituindo em \ref{eq:motion}, temos como as componentes de forças resultantes do sistema a relação \ref{eq:forca3}. 

\begin{equation}
\centering
\begin{pmatrix}
	F_x \\ F_y \\ F_z \\ N_x \\ N_y \\ N_z 
\end{pmatrix} = 
\begin{pmatrix}
-mg\sin{\theta} \\ mg\sin{\phi}\cos{\theta} \\ u - mg\cos{\phi}\cos{\theta} \\ \tau_{\phi} \\ \tau_{\theta} \\ \tau_{\psi} 
\end{pmatrix}
\label{eq:forca3}
\end{equation}

\noindent onde $(mg\sin{\phi},mgsin{\theta},mg\cos{\phi}\cos{\theta})$ são as componentes $x$, $y$ e $z$ do peso do veículo, respectivamente. Por fim, o conjunto de equações que descrevem o movimento do quadricóptero em relação ao referencial do corpo do veículo, com base nas relações \ref{eq:motion} e \ref{eq:forca3} será:

\begin{equation}
\begin{aligned}
	m\left[ { \dot { \upsilon  }  }_{ x } - { \omega  }_{ z }\upsilon_{ y } + { \omega  }_{ y }{ \upsilon  }_{ z } + g\sin{\theta} \right] \quad &=\quad 0 \\ 
	m\left[ { \dot { \upsilon  }  }_{ y } - { \omega  }_{ z }\upsilon_{ x } + { \omega  }_{ x }{ \upsilon  }_{ z } - g\sin{\phi}\cos{\theta} \right] \quad &=\quad 0 \\
	m\left[ { \dot { \upsilon  }  }_{ z } - { \omega  }_{ y }\upsilon_{ x } + { \omega  }_{ x }{ \upsilon  }_{ y } + g\cos{\phi}\cos{\theta} \right] \quad &=\quad u \\ 
	{ I }_{ xx }{ \dot { \omega  }  }_{ x } - (I_{ yy } - I_{ zz }){ \omega  }_{ y }{ \omega  }_{ z }\quad &=\quad \tau_{\phi} \\
	{ I }_{ yy }{ \dot { \omega  }  }_{ y } - (I_{ zz } - I_{ xx }){ \omega  }_{ z }{ \omega  }_{ x }\quad &=\quad \tau_{\theta}\\
	{ I }_{ zz }{ \dot { \omega  }  }_{ z } - (I_{ xx } - I_{ yy }){ \omega  }_{ x }{ \omega  }_{ y }\quad &=\quad \tau_{\psi} 
\end{aligned}
\label{eq:forca4}
\end{equation}

Algumas forças e momentos como efeitos aerodinâmicos, dinâmica dos motores e efeitos giroscópicos são consideradas secundárias e são aqui deixadas de lado, pois:

\begin{itemize}
\item	Para quadricópteros (mini), forças secundárias e momentos de corpos pequenos são dominadas pelas forças principais e vetores de torque, resultando em um pequeno e limitado efeito sobre a dinâmica do veículo.
\item A complexidade de um modelo depende essencialmente da expressão das forças e momentos aerodinâmicos.
\end{itemize}





\newpage

\bibliographystyle{abbrv}
\bibliography{Dissertacao} 

\newpage

\end{document}